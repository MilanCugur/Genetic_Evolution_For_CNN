% Simposium MATHEMATICS and APPLICATIONS
% Requires Latex2e!
\documentclass[eng]{simposium}
\volume{,Vol. X}  %%%%%% TO BE ENTERED BY THE EDITOR(S)
\issue{(1)}      %%%%%% TO BE ENTERED BY THE EDITOR(S)
\pubyear{2019}   %%%%%% TO BE ENTERED BY THE EDITOR(S)
\firstpage{1}    %%%%%% TO BE ENTERED BY THE EDITOR(S)
\lastpage{12}     %%%%%% TO BE ENTERED BY THE EDITOR(S)

%%%%%% ENTER HERE ADDITIONAL PACKAGES
%%%%%% For example: \usepackage{gclc}

%%%%%% ENTER HERE YOUR OWN LATEX COMMANDS
%%%%%% For example: \newcommand{\const}{\mathop{\mathrm{const}}}

\begin{document}
\begin{frontmatter}

\title{Modified hybrid genetic algorithm for training convolutional neural networks}

\author{\textbf{\fnms{Milan M.} \snm{Čugurović}}}
\address{University of Belgrade, Faculty of Mathematics, Studentski trg 16, 11000 Belgrade\\
\email{milan\_cugurovic@matf.bg.ac.rs}
}
\author{\textbf{\fnms{Nikola} \snm{Dimitrijević}}}
\address{Microsoft Development Center Serbia, Španskih boraca 3, 11000 Belgraed\\
\email{nikoladim95@gmail.com}
}
\author{\textbf{\fnms{Stefan} \snm{Mišković}}}
\address{University of Belgrade, Faculty of Mathematics, Studentski trg 16, 11000 Belgrade\\
\email{stefan@matf.bg.ac.rs}
}

\received{\smonth{December} \syear{2019}}   %%%%%% TO BE ENTERED BY THE EDITOR(S)

\maketitle
\begin{abstract}

This paper presents a modified variant of genetic algorithm for training convolutional architectures which reduces the execution time of the algorithm. 
Modification is based on changing the evolutional segment of the algorithm by focusing on limiting the training time of each individual and incorporating the 
learnt knowledge of neuron parameters from the last generation into each new generation. By doing so the evolution is made more efficient, thus reducing the time 
needed to find the desired architecture.

Additional contribution of this paper is creating new dataset \textit{DoubledMNIST}, which represents a successor of the popular MNIST dataset.
Created dataset is doubled with respect to the MNIST dataset both in terms of the number of instances and in terms of the resolution of each individual isntance.
Results shown in the paper were obtained using the presented improved method on the created dataset. The paper also shows classification results on the said dataset.
\end{abstract}

\begin{keyword}
genetic algorithm; local search; convolutional arhitectures; MNIST dataset
\end{keyword}
\end{frontmatter}

\section{Algorithm}
% da li je ok da citiramo sa arxiva
% \received i slicno

The core idea of the genetic algorithm is to get a good solution to a problem by generating better and better solutions through the process of evolution.
The evolution process consists of selection, mutation and crossover
\marginpar{test}

\begin{thebibliography}{99}
\bibitem{gen}   
\textbf{Xie L and Yuille A.} Genetic CNN. \emph{arXiv preprint arXiv:1703.01513.}, 2017.

\bibitem{neuroevo} 
\textbf{Floreano, D., Dürr, P., Mattiussi, C.} Neuroevolution: from architectures to learning. \emph{Evolutionary intelligence}, 1(1), 47-62, 2008.

\bibitem{meta} 
\textbf{Voß, S., Martello, S., Osman, I. H., Roucairol, C. (Eds.).} Meta-heuristics: Advances and trends in local search paradigms for optimization. \emph{Springer Science and Business Media.}, 2012.

\bibitem{emnist}   
\textbf{Cohen, G., Afshar, S., Tapson, J., van Schaik, A.} EMNIST: an extension of MNIST to handwritten letters. \emph{arXiv preprint arXiv:1702.05373.}, 2017.

\end{thebibliography}

\end{document}
